% ---------- SECTION I - INTRODUCTION ----------

\section{Introduction}
\label{sec:intro}

\lettrine[nindent=0em,lines=3]{P}asswords are everywhere in the daily online life of almost every person. No matter if it is a new social media platform, an e-commerce website or online banking - dozens of services require internet users to authenticate using a, preferably strong and unique, password \cite{nist}.\\
As many of these accounts contain very sensitive data, users are afraid of their credentials becoming exposed during one of the many data breaches of the recent years \cite{statista_dossier2018}. Monitoring services like Have I Been Pwned\footnote{https://haveibeenpwned.com} or the Hasso-Plattner-Institut Identity Leak Checker\footnote{https://sec.hpi.de/ilc/} try to gather information about those breaches, and to inform affected internet users. The former service currently reports almost ten billion breached accounts from 440 different websites, the latter over ten billion from over 1,000 leaks \cite{hibp,hpi}.\\
A known problem of those breaches is that many people reuse their mostly weak passwords \cite{bailey2014} - around 70\% of the credentials in a new breach published in \cite{hibp} are already in the system. Some websites are trying to get around this problem by enforcing arbitrary password requirements like length, use of special characters or banning known weak passwords. Oftentimes those measures do not increase password security, but do confuse the users trying to generate valid credentials for a new website \cite{hunt2017}; therefore, the American \ac{nist} does no longer recommend using such rules \cite{nist}.\\
Although password managers provide an effective way of generating and storing long, random and therefore unique passwords, they introduce an additional step in the authentication flow \cite{lyastani2018}.
But even those do not protect users against phishing or misconfigured services that store the password in plain text instead of hashed and salted, or accidentally log it like Facebook did until 2019 \cite{gallagher2019}.\\
After all, most security researchers as well as empirical evidence agree that passwords are a less than optimal form of authentication, especially because their security relies on the choices users make - and we know that those are more often than not weak ones \cite{hunt2018c,whitty2015}.\\
\\
Is a secure and easy-to-use authentication possible without passwords or even storing any credentials?\\
The FIDO Alliance, a project supported by hundreds of tech giants like Amazon, Google, Facebook, Apple, American Express, Paypal and many others\footnote{https://fidoalliance.org/members/} claims to have a solution: The \ac{fido2} authentication framework. Logging in would no longer require a password\footnote{\ac{fido2} can also be used as an additional factor, see chapter \ref{sec:foundations}.}, but a hardware token (\emph{authenticator}) like a USB stick. Because the technology utilizes public-key cryptography, the actual secret would never leave the authenticator, drastically improving security.\\
This paper aims to summarize the state of the art of research about \ac{fido2} and to answer the following research question:

\begin{displayquote}
    \textbf{R:} \emph{Can the FIDO2 framework provide secure and easy-to-use web authentication?}
\end{displayquote}

\noindent Considering the potential impact of weak, reused and leaked passwords, and the recent support of \ac{fido2} by many companies with the goal to get rid of passwords \cite{ng2019,mingis2020,mehta2018,fido2_overview}, a closer look at the current research regarding this topic and possible open questions is relevant for both academics and practitioners.\\
Previous literature has already shown evaluations of many different authentication methods, including \acp{fido2} predecessors \ac{u2f} and \ac{uaf}, coming to rather depressing conclusions \cite{bonneau2012,hunt2018a,lang2017,das2018} - although passwords are known to be insecure and prone to a variety of attacks, most other options are considered too complex to either implement or use.