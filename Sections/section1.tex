% - 80% of leaked passwords weak
% - rate of leaks increases
% - More and more accounts per user
% - Similar, weak passwords vs. usability of unique, safe passwords
% - FIDO based on public key cryptography -> the actual credential is never stored
% - Goal is cross-platform auth
%  - Built-In Auth to Android/Windows (Win Hello)

\section{Introduction}
\label{sec:intro}

\lettrine[nindent=0em,lines=3]{T}est.

% - Passwords are the prevalent authentication mechanism
% - Authentication is base for establishing the digital identity \cite{nist}
% - Users have more and more accounts
% - Despite better knowledge, users reuse their (already weak) passwords \cite{bailey2014,hunt2018c} -> Up to 70\% of passwords in a breach where already known
% - HIBP has almost 10bn accounts from 440 websites \cite{hibp}, HPI more than 10bn accounts from over 1,000 leaks \cite{hpi} (top 4 sites ~49% of accounts from mailing services)
% - Security guidelines and password requirements confuse people, but do not lead to better passwords \cite{hunt2017}
% - Password managers solve some of these problems \cite{lyastani2018}, but also provide a barrier
% - Phishing and data breaches are still possible
% - How can authentication be possible without passwords or even storing credentials?
% - Microsoft, Google, Apple and many others push FIDO2 \cite{ng2019,mingis2020,mehta2018,fido2_overview}
% - Public-Key-Cryptography and biometrics for better usability \cite{dunkelberger2018}

\begin{displayquote}
    \textbf{R:} \emph{Can the FIDO2 framework provide secure and easy-to-use web authentication?}
\end{displayquote}