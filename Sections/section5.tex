% ---------- SECTION V - Discussion ----------

\section{Discussion}
\label{sec:discussion}

As in the previous chapter, the evaluation of the results is also split into the two dimensions usability and security.
Looking at the results, the research question probably has to be answered with \emph{Yes, but\dots}.\\
In general, \ac{fido2} can indeed provide secure and easy-to-use authentication. In fact, the comparison in figure \ref{fig:bonneau_matrix} shows that it is significantly more secure than using passwords. From a usability point of view passwords seem to be better in two, but worse in three categories. Especially the memory effort is a key factor in why passwords tend to be insecure, because users can't memorize too many different ones \cite{lyastani2018,elhai2016,whitty2015}.\\
\\
The results also clearly show that the FIDO Alliance has to provide better answers to the problem of account recovery than to recommend using a second key for each site. Having a standardized process or even a way of revoking the lost key from all accounts at once would probably have a huge impact on usability and therefore acceptance.\\
Some other challenges rely heavily on the relying party and how they implement the application. As shown by \cite{das2018}, having clear and easy to understand instructions could drastically decrease the complexity of the setup. Account delegation might also be possible by using additional, password-like authentication tokens. Such tokens could be randomly generated by the main user inside the application, and then distributed like regular passwords. Although this would increase the attack surface and re-introduce security problems of passwords, the risk could be minimized by limiting the token to either a time frame (e.g. valid for 24h) or a number of logins allowed.\\
What remains is mainly the issue of having to carry around a physical device. With support of \ac{fido2} in Android and soon iOS, using a smartphone as an authenticator reduces that problem, as many people already carry their phone at all times. This, however, would decrease the compatibility, as the phone would require a connection using Bluetooth or \ac{nfc} due to the lack of a standrad USB-A port.

% ---- Subsection Threats to Validity ----
\subsection{Threats to Validity}
\label{subsec:validity_threats}

As this work is mostly a literature review, the limitations of the results above are mainly dependent on said literature.\\

As seen in table \ref{tab:literature_review}, the first phase returned most of the FIDO2 sources. This is due to the fact that these are mostly media outlets like technical blogs \citep{hunt2018b,leitner2019, chonng2018, ng2019, mingis2020} or from the official FIDO Alliance website \citep{fido2_overview,fido2_webauthn}.

\begin{table}[ht]
    \centering
    \caption{Overview of literature used in this paper.}
    \label{tab:literature_review}
    \begin{tabular}{ l | c | c }
        \textbf{Phase} & \textbf{Foundations} & \textbf{FIDO Framework}\\
        \hline
        Phase 1 & 5 & 7\\
        Phase 2 & 6 & 6\\
        Phase 3 & 4 & 2\\
    \end{tabular}
\end{table}

\noindent As of today, there is rather little scientific research regarding \ac{fido2} specifically. In fact, most of the citations on \ac{fido2} are either technical documentation (provided by the FIDO Alliance), whitepapers or blog posts by security professionals. Although these sources do not necessarily follow the scientific method, they are still very useful in getting insights into how \ac{fido2} works.\\
It is for the same reason that only the conceptual security of the whole framework could be evaluated, those concepts however are based on mature and well-tested techniques like public-key cryptography and challenge-response authentication, so the results on these parts is still sound.\\
The main part of scientific research focuses on the usability or acceptability of the system.