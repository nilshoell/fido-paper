% ---------- SECTION II - METHODS ----------

\section{Methods}
\label{sec:methods}

To get detailed information about the current body of knowledge regarding \ac{fido2} and to answer the research question, a three-phased literature review is conducted.\\
The first phase is an unstructured internet search using common search engines like Google\footnote{https://google.com} or DuckDuckGo\footnote{https://duckduckgo.com/}. The aim of these searches is to get a basic understanding of what \ac{fido2} is, how it works, and how it is perceived in different media outlets.\\
These findings enable the second phase, a structured search for scientific literature regarding different subtopics or aspects of the FIDO-Framework, password-based authentication and other foundations. It makes use of more research-oriented search engines like Google Scholar\footnote{https://scholar.google.com/}, ScienceDirect\footnote{https://www.sciencedirect.com/}, IEEE Xplore\footnote{https://ieeexplore.ieee.org/}, AISeL\footnote{https://aisel.aisnet.org/} and the official FIDO Alliance whitepaper page\footnote{https://fidoalliance.org/white-papers/}.\\
In the last phase a search for secondary literature is conducted in the results of phase two, aiming for related scientific research on the topic.\\
\\
To answer the research question the framework proposed by Bonneau et al. (2012) \cite{bonneau2012} is used as a base, as the different benefits used for evaluating authentication methods are often referred in literature. However, the dimension \emph{Deployability} is dropped, leaving a focus on \emph{Usability} and \emph{Security} as proposed in the research question.\\
Those two parts are again supported by using other evaluation frameworks. For usability this will be the categorization proposed by Nielsen (1993) to make sure all aspects of acceptability and usability are covered. The security section is backed by a guide from the \ac{owasp}, more specifically the second vulnerability in the 2017 edition of \acp{owasp} Top 10, \emph{Broken Authentication}.