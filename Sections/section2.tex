% ---------- SECTION II - METHODS ----------

\section{Methods}
\label{sec:methods}

% - First unstructured internet research
% - More specific research for scientific literature for certain aspects
% - From there on secondary literature

To get detailed information about the state of the art of research about \ac{fido2} and to answer the research question, a three-phased literature review is conducted.\\
The first phase is an unstructured internet search using common search engines like Google\footnote{https://google.com} or DuckDuckGo\footnote{https://duckduckgo.com/}. The aim of these searches is to get a basic understanding of what \ac{fido2} is, how it works, and how it is perceived in different media outlets.\\
These findings enable the second phase, a structured search for scientific literature regarding different subtopics or aspects of the FIDO-Framework, password-based authentication and other foundations. It makes use of more research-oriented search engines like Google Scholar\footnote{https://scholar.google.com/}, ScienceDirect\footnote{https://www.sciencedirect.com/}, IEEE Xplore\footnote{https://ieeexplore.ieee.org/}, AISeL\footnote{https://aisel.aisnet.org/} and the official FIDO Alliance whitepaper page\footnote{https://fidoalliance.org/white-papers/}.\\
In the last phase a search for secondary literature is conducted in the results of phase two, aiming for related scientific research on the topic.\\
\\
As seen in table \ref{tab:literature_review}, the first phase returned most of the FIDO2 sources. This is due to the fact, that these are mostly media outlets like technical blogs \citep{hunt2018b,leitner2019, chonng2018, ng2019, mingis2020} or from the official FIDO Alliance website \citep{fido2_overview,fido2_webauthn}.

\begin{table}[ht]
    \centering
    \caption{Overview of literature used in this paper.}
    \label{tab:literature_review}
    \begin{tabular}{ l | c | c }
        \textbf{Phase} & \textbf{Foundations} & \textbf{FIDO Framework}\\
        \hline
        Phase 1 & 5 & 7\\
        Phase 2 & 6 & 6\\
        Phase 3 & 4 & 2\\
        % Phase 1 & \cite{hunt2011,hunt2018a,hpi,hibp,gallagher2019} & \cite{fido2_overview,fido2_webauthn,hunt2018b,leitner2019, chonng2018, ng2019, mingis2020, fido2_ctap}\\
        % Phase 2 & \cite{nist,bailey2014,elhai2016,whitty2015,turner2016,platt2015} & \cite{mdn_webauthn,lyastani2020,ehta2018,dunkelberger2018,gomi2019,statista_2fa}\\
        % Phase 3 & \cite{hunt2017,hunt2018c,lyastani2018,bonneau2012} & \cite{lang2017,das2018}\\
    \end{tabular}
\end{table}

To answer the research question the framework proposed by Bonneau et al \citep{bonneau2012} is used as a base, as the different benefits used for evaluating authentication methods are often referred in literature. However, the dimension \emph{Deployability} is dropped, leaving a focus on \emph{Usability} and \emph{Security}.