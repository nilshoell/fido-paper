% ---------- SECTION IV - FIDO2 & WebAuthn ----------

\section{Results}
\label{sec:results}

% In this chapter
% J. Bonneau, C. Herley, P. C. v. Oorschot, and F. Stajano, “The quest to replace passwords"

% ---- Subsection Usability ----
\subsection{Usability}
\label{subsec:usability}

% - Users accept 1FA security tokens
% - Hard/Impossible to use with public computers (no way to insert authenticator)
% - Impossible to delegate access to trusted persons
% - Loosing/Destroying Key -> no access, complicated recovery
%  - Register 2 keys (see: Google Advanced Protection\footnote{https://landing.google.com/advancedprotection/})\cite{gomi2019} or re-run identity checks
% - Many studies for U2F: 25, 26, 39, 40, 41
%  - J. Lang, A. Czeskis, D. Balfanz, M. Schilder, and S. Srinivas, “Security keys: Practical cryptographic second factors for the modern web”
%  - S. Das, G. Russo, A. C. Dingman, J. Dev, O. Kenny, and L. J. Camp,“A qualitative study on usability and acceptability of yubico security key”
%  - S. Das, A. Dingman, and L. J. Camp, “Why johnny doesn’t use two factor: A two-phase usability study of the fido u2f security key”
%  - J. Reynolds, T. Smith, K. Reese, L. Dickinson, S. Ruoti, and K. Seamons, “A tale of two studies: The best and worst of yubikey usability”
% - Rollo-Out in professional environment (Google, 50,000 users) as 2F very successful \cite{lang2017}
% - Response to Key very different across demographics \cite{das2018}, confusing concept, esp. unclear instructions
% - Clear setup instructions needed
% - Form factor especially relevant for older generation


% ---- Subsection Security ----
\subsection{Security}
\label{subsec:security}

% - Public-Key-Cryptography
% - No Phishing, Replay or data breaches possible
% - Keys are Read-Only
% - Require physical presence or even PIN/Biometrics
% - NO resilience to theft
% -> Hardware key can be secured using PIN/Fingerprint, adding knowledge/inheritance factor


% ---- Subsection Other ----
\subsection{Challenges}
\label{subsec:challenges}

% - Users want to revoke keys -> loss of key associated with loss of account
% - Completely different approach, breaking changes -> Users cannot evaluate the security of this concepts