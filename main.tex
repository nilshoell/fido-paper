%----------------------------------------------------------------------------------------
%	PAPER SETUP
%----------------------------------------------------------------------------------------

\newcommand*{\mytitle}{Beyond Passwords? About the current state of FIDO2 authentication} % Paper Title
\newcommand*{\myauthor}{Nils Höll} % Paper Author
\newcommand*{\mydate}{May 2020} % Publication Date
\newcommand*{\myuni}{\href{https://www.uni-due.de}{University of Duisburg-Essen}} % University
\newcommand*{\myfac}{\href{https://www.wiwi.uni-due.de}{Faculty of Business Administration and Economics}} % Faculty
\newcommand*{\myorg}{ACWE} % Organization, Journal, Subject this paper was published in
\newcommand*{\mysup}{Peter M. Schuler} % Supervisor


%----------------------------------------------------------------------------------------
%	PACKAGES AND OTHER DOCUMENT CONFIGURATIONS
%----------------------------------------------------------------------------------------

\documentclass[twoside,twocolumn]{article}

\usepackage{blindtext} % Package to generate dummy text throughout this template 

% ------- FONTS -------
\usepackage[sc]{mathpazo} % Use the Palatino font
\usepackage[T1]{fontenc} % Use 8-bit encoding that has 256 glyphs
\usepackage[utf8]{inputenc}

\usepackage[official]{eurosym}
\DeclareUnicodeCharacter{20AC}{\euro{}}

\linespread{1.05} % Line spacing - Palatino needs more space between lines
\usepackage{microtype} % Slightly tweak font spacing for aesthetics

\usepackage[english]{babel} % Language hyphenation and typographical rules

% ------- DOCUMENT MARGINS -------
\usepackage[hmarginratio=1:1,top=32mm,columnsep=20pt]{geometry} % Document margins
\usepackage[hang, small,labelfont=bf,up,textfont=it,up]{caption} % Custom captions under/above floats in tables or figures
\usepackage{booktabs} % Horizontal rules in tables

\usepackage{graphicx}
% \usepackage[section]{placeins} % Floats are always placed in the right section
\usepackage{lettrine} % The lettrine is the first enlarged letter at the beginning of the text

% ------- LISTS -------
\usepackage{enumitem} % Customized lists
\setlist[itemize]{noitemsep} % Make itemize lists more compact

% ------- ABSTRACT -------
\usepackage{abstract} % Allows abstract customization
\renewcommand{\abstractnamefont}{\normalfont\bfseries} % Set the "Abstract" text to bold
\renewcommand{\abstracttextfont}{\normalfont\small\itshape} % Set the abstract itself to small italic text

% ------- SECTION TITLES -------
\usepackage{titlesec} % Allows customization of titles
\renewcommand\thesection{\Roman{section}} % Roman numerals for the sections
\renewcommand\thesubsection{\roman{subsection}} % roman numerals for subsections
\titleformat{\section}[block]{\large\scshape\centering}{\thesection.}{1em}{} % Change the look of the section titles
\titleformat{\subsection}[block]{\large}{\thesubsection.}{1em}{} % Change the look of the section titles

% ------- HEADERS AND FOOTERS -------
\usepackage{fancyhdr} % Headers and footers
\pagestyle{fancy} % All pages have headers and footers
\fancyhead{} % Blank out the default header
\fancyfoot{} % Blank out the default footer
\fancyhead[C]{\mytitle\, $\bullet$ \mydate\, $\bullet$ \myorg} % Custom header text
\fancyfoot[RO,LE]{\thepage} % Custom footer text

\usepackage{titling} % Customizing the title section

% ------- HYPERREF -------
\usepackage{hyperref} % For hyperlinks in the PDF
\AtBeginDocument{
  \hypersetup{pdftitle={\mytitle}} % Set the PDF's title to your title
  \hypersetup{pdfauthor={\myauthor}} % Set the PDF's author to your name
  \hypersetup{hidelinks} % Prints all links black; comment out for default LaTeX behavior
}

% ------- BIBTEX -------
\usepackage[backend=bibtex,style=numeric,citestyle=numeric,natbib=true]{biblatex}
\addbibresource{sources.bib}

\usepackage[autostyle=true]{csquotes}

% ------- COLORS -------
% Define colors for highlighting
\usepackage{xcolor}
\definecolor{lstbg}{gray}{0.95}
\definecolor{lstComment}{RGB}{51, 102, 0}
\definecolor{lstKey}{RGB}{0, 51, 204}
\definecolor{lstStr}{RGB}{162, 43, 43}

% ------- ACRONYMS -------
% \usepackage[nohyperlinks]{acronym} % Prints only used acronyms in overview
\usepackage[nohyperlinks, printonlyused]{acronym} % Prints only used acronyms in overview

% Better highlighting for inline code
\newcommand{\linecode}[1]{%
  \colorbox{lstbg}{\textcolor{lstStr}{\textbf{\texttt{#1}}}}%
}

%----------------------------------------------------------------------------------------
%	TITLE SECTION
%----------------------------------------------------------------------------------------

\setlength{\droptitle}{-2\baselineskip} % Move the title up

\pretitle{%
  \begin{center}
    \Large\myuni\\
    \large\myfac\\
    \vspace{0.5cm}
    \Huge\bfseries
    } % Article title formatting
\posttitle{\vspace{1.5cm}\end{center}} % Article title closing formatting
\title{\mytitle} % Article title
\author{%
  \textsc{\myauthor} \\[1ex] % Name
  \normalsize 3087271 \\ % MatriculationNo
  \normalsize \href{mailto:nils.siegle@stud.uni-due.de}{nils.siegle@stud.uni-due.de} \\[2ex]
  \normalsize \emph{Supervisor:} \mysup \\
  [5ex]
  \mydate
}
\date{} % Leave empty to omit a date
\renewcommand{\maketitlehookd}{%

%------------------------------------------------
%	ABSTRACT
%------------------------------------------------

\begin{abstract}
\noindent Passwords are insecure and annoying to use, especially if one tries to use them in a secure way. We know the problems regarding this form of authentication, there are dozens of studies and articles about why passwords should not be used anymore, how they can be made more secure, and why people still reuse already weak credentials despite better knowledge \cite{bailey2014,elhai2016,hunt2018c,whitty2015,mcmillan2012}.\\
Those findings and their conclusions - that we need better forms of authentication for web applications - are supported by the almost regular credential leaks from companies in all branches, collected by sites like Have I Been Pwned or the Hasso-Plattner-Institut Identity Leak Checker.\\
One of the many proposals on how to tackle this problem comes from the \ac{fido} Alliance. Their new authentication framework, \ac{fido2}, promises an open standard for secure and easy to use web authentication.\\
\\
This paper analyzes the current body of knowledge regarding the security and usability of \ac{fido2} and tries to draw a conclusion whether or not it could replace legacy passwords in the future.
\end{abstract}
}

%----------------------------------------------------------------------------------------

\begin{document}

% Print the title
\maketitle

\newpage
\onecolumn
%------------------------------------------------
%	ToC
%------------------------------------------------
\tableofcontents
\newpage

%------------------------------------------------
%	ACRONYMS
%------------------------------------------------

\section*{Abbreviations}
\begin{acronym}[AAAAAAA]
  \acro{1fa}[1FA]{Single-Factor Authentication}
  \acro{2fa}[2FA]{Two-Factor Authentication}
  \acro{authn}[authn]{Authentication}
  \acro{authz}[authz]{Authorization}
  \acro{api}[API]{Application Programming Interface}
  \acro{ctap}[CTAP]{Client to Authenticator Protocol}
  \acro{ctap2}[CTAP2]{Client to Authenticator Protocol 2}
  \acro{fido}[FIDO]{Fast IDentity Online}
  \acro{fido2}[FIDO2]{Fast IDentity Online 2}
  \acro{hotp}[HOTP]{HMAC-based One-time Password}
  \acro{http}[HTTP]{Hypertext Transport Protocol}
  \acro{https}[HTTPS]{Hypertext Transport Protocol Secure}
  \acro{it}[IT]{Information Technology}
  \acro{mfa}[MFA]{Multi-Factor Authentication}
  \acro{nist}[NIST]{National Institute of Standards and Technology}
  \acro{nfc}[NFC]{Near-Field Communication}
  \acro{os}[OS]{Operating System}
  \acro{otp}[OTP]{One-time Password}
  \acro{owasp}[OWASP]{Open Web Application Security Project}
  \acro{pin}[PIN]{Personal Identification Number}
  \acro{rp}[RP]{Relying Party}
  \acro{tpm}[TPM]{Trusted Platform Module}
  \acro{totp}[TOTP]{Time-based One-time Password}
  \acro{u2f}[U2F]{Universal Second Factor}
  \acro{uaf}[UAF]{Universal Authentication Framework}
  \acro{w3c}[W3C]{World Wide Web Consortium}
\end{acronym}

% \newacro{id}[ABRV]{Abbreviation}
% \acro{}[]{}
% Use 
% - \ac{id} for standard behavior
% - \acs{id} for acronym
% - \acl{id} for long version
% - \acp{id} for plural (with 's' at the end)

\newpage
\twocolumn

%----------------------------------------------------------------------------------------
%	ARTICLE CONTENTS
%----------------------------------------------------------------------------------------

% Section 1 - Introduction
\section{Introduction}
\label{sec:intro}

\lettrine[nindent=0em,lines=3]{S}tack Overflow is a renowned online community amongst enthusiastic programmers that share their knowledge or are looking to improve it using the question and answer platform.
More than 19 million questions have been asked so far and 70\% of them have been answered \cite{stackExchangeQuery}. Although some of the questions might remain unanswered, others receive a solution quite fast.
So, what is making the difference? This was the starting point when our research question took shape:

\begin{displayquote}
    \textbf{R:} \emph{What features of a question have an influence on the answer time?}
\end{displayquote}

\blindtext


% Section 2 - Foundations
% ---------- SECTION II - METHODS ----------

\section{Methods}
\label{sec:methods}

% - First unstructured internet research
% - More specific research for scientific literature for certain aspects
% - From there on secondary literature

To get detailed information about the current body of knowledge regarding \ac{fido2} and to answer the research question, a three-phased literature review is conducted.\\
The first phase is an unstructured internet search using common search engines like Google\footnote{https://google.com} or DuckDuckGo\footnote{https://duckduckgo.com/}. The aim of these searches is to get a basic understanding of what \ac{fido2} is, how it works, and how it is perceived in different media outlets.\\
These findings enable the second phase, a structured search for scientific literature regarding different subtopics or aspects of the FIDO-Framework, password-based authentication and other foundations. It makes use of more research-oriented search engines like Google Scholar\footnote{https://scholar.google.com/}, ScienceDirect\footnote{https://www.sciencedirect.com/}, IEEE Xplore\footnote{https://ieeexplore.ieee.org/}, AISeL\footnote{https://aisel.aisnet.org/} and the official FIDO Alliance whitepaper page\footnote{https://fidoalliance.org/white-papers/}.\\
In the last phase a search for secondary literature is conducted in the results of phase two, aiming for related scientific research on the topic.\\
\\
As seen in table \ref{tab:literature_review}, the first phase returned most of the FIDO2 sources. This is due to the fact that these are mostly media outlets like technical blogs \citep{hunt2018b,leitner2019, chonng2018, ng2019, mingis2020} or from the official FIDO Alliance website \citep{fido2_overview,fido2_webauthn}.

\begin{table}[ht]
    \centering
    \caption{Overview of literature used in this paper.}
    \label{tab:literature_review}
    \begin{tabular}{ l | c | c }
        \textbf{Phase} & \textbf{Foundations} & \textbf{FIDO Framework}\\
        \hline
        Phase 1 & 5 & 7\\
        Phase 2 & 6 & 6\\
        Phase 3 & 4 & 2\\
        % Phase 1 & \cite{hunt2011,hunt2018a,hpi,hibp,gallagher2019} & \cite{fido2_overview,fido2_webauthn,hunt2018b,leitner2019, chonng2018, ng2019, mingis2020, fido2_ctap}\\
        % Phase 2 & \cite{nist,bailey2014,elhai2016,whitty2015,turner2016,platt2015} & \cite{mdn_webauthn,lyastani2020,ehta2018,dunkelberger2018,gomi2019,statista_2fa}\\
        % Phase 3 & \cite{hunt2017,hunt2018c,lyastani2018,bonneau2012} & \cite{lang2017,das2018}\\
    \end{tabular}
\end{table}

\noindent The rather small amount of actual scientific research found in the phases two and three already shows a need for further work in this field.\\
\\
To answer the research question the framework proposed by Bonneau et al. (2012) \cite{bonneau2012} is used as a base, as the different benefits used for evaluating authentication methods are often referred in literature. However, the dimension \emph{Deployability} is dropped, leaving a focus on \emph{Usability} and \emph{Security} as proposed in the research question.\\
Those two parts are again supported by using other evaluation frameworks. For usability this will be the categorization proposed by Nielsen (1993) to make sure all aspects of acceptability and usability are covered. The security section is backed by a guide from the \ac{owasp}, more specifically the second vulnerability in the 2017 edition of \acp{owasp} Top 10, \emph{Broken Authentication}.

% Section 3 - Methods
% ---------- SECTION III - FOUNDATIONS ----------
% - [] Authentication and Authorisation
% - [] Password-based auth

\section{Foundations}
\label{sec:foundations}

The following chapter explains some basic concepts of web authentication and digital identities.

% ---- Authn and Authz ----
\subsection{Authentication and Authorisation}
\label{subsec:authn_authz}

Authentication and authorisation, in technical environments often referred to as \ac{authn} and \ac{authz} are concepts present in virtually every protected \ac{it} ressource.\\
The formers describes the process of identifying a user by verifying their digital identity, hence checking their \emph{authenticity}. Oftentimes only certain users are \emph{authorized} to access a protected ressource (e.g. specific parts of a website, sensitive data etc.). Therefore, an authorization has to take place both to ensure that the correct information is distributed to each user, and that no one without permission is able to manipulate data.

% ---- Password-Based ----
\subsection{Password-Based Authentication}
\label{subsec:pw_based_authn}

The most common and widespread type of web authentication is the so-called password-based authentication. In this case a user has to provide a \emph{username} (oftentimes an e-mail address, as these are globally unique) and a \emph{password} or \emph{passphrase}, a secret string only known to the user.\\
Only the correct combination of username and password grants access to the protected resource.

% ---- MFA ----
\subsection{Multi-Factor Authentication}
\label{subsec:mfa}

% - Often \ac{2fa} using \ac{hotp} or \ac{totp}
% - Also possible with key files
% - Only guessing or stealing the password is not enough


% ---- Federated Auth ----
\subsection{Federated Authentication}
\label{subsec:fed_auth}

% - Federated Identity Providers
% - Shibboleth, Kerberos
% - OAuth, OpenID, Login with Google/Facebook
% - Target resource (SP) never knows the password
% - User only has to remember a single password for many services, without increased risk


% Section 4 - Results
% ---------- SECTION IV - FIDO2 & WebAuthn ----------

\section{Results}
\label{sec:results}

% In this chapter
% J. Bonneau, C. Herley, P. C. v. Oorschot, and F. Stajano, “The quest to replace passwords"

% ---- Subsection Usability ----
\subsection{Usability}
\label{subsec:usability}

% - Users accept 1FA security tokens
% - Hard/Impossible to use with public computers (no way to insert authenticator)
% - Impossible to delegate access to trusted persons
% - Loosing/Destroying Key -> no access, complicated recovery
%  - Register 2 keys (see: Google Advanced Protection)
% - Many studies for U2F: 25, 26, 39, 40, 41
%  - J. Lang, A. Czeskis, D. Balfanz, M. Schilder, and S. Srinivas, “Security keys: Practical cryptographic second factors for the modern web”
%  - S. Das, G. Russo, A. C. Dingman, J. Dev, O. Kenny, and L. J. Camp,“A qualitative study on usability and acceptability of yubico security key”
%  - S. Das, A. Dingman, and L. J. Camp, “Why johnny doesn’t use two factor: A two-phase usability study of the fido u2f security key”
%  - J. Reynolds, T. Smith, K. Reese, L. Dickinson, S. Ruoti, and K. Seamons, “A tale of two studies: The best and worst of yubikey usability”
%  - 
% - Clear setup instructions needed
% - Form factor especially relevant for older generation


% ---- Subsection Security ----
\subsection{Security}
\label{subsec:security}

% - Public-Key-Cryptography
% - No Phishing, Replay or data breaches possible
% - Keys are Read-Only
% - Require physical presence or even PIN/Biometrics
% - NO resilience to theft


% ---- Subsection Other ----
\subsection{Challenges}
\label{subsec:challenges}

% - Users want to revoke keys -> loss of key associated with loss of account
% - Completely different approach, breaking changes -> Users cannot evaluate the security of this concepts

% Section 5 - Discussion
% ---------- SECTION V - Discussion ----------

\section{Discussion}
\label{sec:discussion}

% - Problem of recovery after loss
% - Good communication of (security) benefits and instructions needed


% ---- Subsection Threats to Validity ----
\subsection{Threats to Validity}
\label{subsec:validity_threats}

% - Little scientific research to this date
% - Rather small and either academic or technical participants in the studies

% Section 6 - Conclusion
% ---------- SECTION VI - CONCLUSION ----------

\section{Conclusion}
\label{sec:conclusion}


% Single column layout for the rest of the paper
\onecolumn

%----------------------------------------------------------------------------------------
%  BIBLIOGRAPHY
%----------------------------------------------------------------------------------------
\printbibliography[heading=bibintoc]

%------------------------------------------------

% Section 5 - Appendix
% % ---------- SECTION VII - APPENDIX ----------

\section{Appendix}
\label{sec:appendix}

% ---- Appendix I ----
\subsection{Appendix I}
\label{apx_1}


%----------------------------------------------------------------------------------------

\end{document}
